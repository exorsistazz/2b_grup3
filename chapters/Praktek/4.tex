\section{Rangga Putra Ramdani}
\subsection{Soal 1}
Berikut adalah pemanggilan file csv dengan library csv yang menggunakan list
\lstinputlisting[firstline=10, lastline=20]{src/4/1174056/praktek/c_1174056_csv.py}

\subsection{Soal 2}
Berikut adalah pemanggilan file csv dengan library csv yang menggunakan dictionary
\lstinputlisting[firstline=22, lastline=31]{src/4/1174056/praktek/c_1174056_csv.py}

\subsection{Soal 3}
Berikut adalah pemanggilan file csv dengan library pandas yang menggunakan list
\lstinputlisting[firstline=9, lastline=11]{src/4/1174056/praktek/p_1174056_pandas.py}

\subsection{Soal 4}
Berikut adalah pemanggilan file csv dengan library pandas yang menggunakan dictionary
\lstinputlisting[firstline=13, lastline=16]{src/4/1174056/praktek/p_1174056_pandas.py}

\subsection{Soal 5}
Berikut penggunaan untuk merubah standar penulisan tanggal, yang mengikuti standar penulisan dari pandas.
\lstinputlisting[firstline=18, lastline=20]{src/4/1174056/praktek/p_1174056_pandas.py}

\subsection{Soal 6}
Berikut merupakan pergantian index kolom
\lstinputlisting[firstline=22, lastline=24]{src/4/1174056/praktek/p_1174056_pandas.py}

\subsection{Soal 7}
berikut merupakan penggunaan untuk merename atribut yang digunakan, atau merubah nama header 0
\lstinputlisting[firstline=26, lastline=30]{src/4/1174056/praktek/p_1174056_pandas.py}

\subsection{Soal 8}
\lstinputlisting[firstline=8, lastline=10]{src/4/1174056/praktek/main_shasa.py}

\subsection{Soal 9}
\lstinputlisting[firstline=11, lastline=14]{src/4/1174056/praktek/main_shasa.py}

\subsection{Penanganan Error}
Dalam praktek kali ini alhamdulliah tidak menemukan error


\section{Teddy Gideon Manik}
\subsection{Soal 1}
Berikut adalah pemanggilan file csv dengan library csv yang menggunakan list
\lstinputlisting[firstline=10, lastline=20]{src/4/1174038/praktek/c_1174038_csv.py}

\subsection{Soal 2}
Berikut adalah pemanggilan file csv dengan library csv yang menggunakan dictionary
\lstinputlisting[firstline=22, lastline=31]{src/4/1174038/praktek/c_1174038_csv.py}

\subsection{Soal 3}
Berikut adalah pemanggilan file csv dengan library pandas yang menggunakan list
\lstinputlisting[firstline=9, lastline=11]{src/4/1174038/praktek/p_1174038_pandas.py}

\subsection{Soal 4}
Berikut adalah pemanggilan file csv dengan library pandas yang menggunakan dictionary
\lstinputlisting[firstline=13, lastline=16]{src/4/1174038/praktek/p_1174038_pandas.py}

\subsection{Soal 5}
Berikut penggunaan untuk merubah standar penulisan tanggal, yang mengikuti standar penulisan dari pandas.
\lstinputlisting[firstline=18, lastline=20]{src/4/1174038/praktek/p_1174038_pandas.py}

\subsection{Soal 6}
Berikut merupakan pergantian index kolom
\lstinputlisting[firstline=22, lastline=24]{src/4/1174038/praktek/p_1174038_pandas.py}

\subsection{Soal 7}
berikut merupakan penggunaan untuk merename atribut yang digunakan, atau merubah nama header 0
\lstinputlisting[firstline=26, lastline=30]{src/4/1174038/praktek/p_1174038_pandas.py}

\subsection{Soal 8}
\lstinputlisting[firstline=8, lastline=10]{src/4/1174038/praktek/main_teddy.py}

\subsection{Soal 9}
\lstinputlisting[firstline=11, lastline=14]{src/4/1174038/praktek/main_teddy.py}

\subsection{Penanganan Error}
Dalam praktek kali ini alhamdulliah tidak menemukan error

\section{Harun Ar-Rasyid}
\subsection{Soal 1}
Isi jawaban soal ke-1

Kalau mau dibikin paragrap \textbf{cukup enter aja}, tidak usah pakai \verb|par| dsb

%\subsection{Soal 2}
%Isi jawaban soal ke-2

%\subsection{Soal 3}
%Isi jawaban soal ke-3

\section{Sri Rahayu}
\subsection{Soal 1}
Isi jawaban soal ke-1

Kalau mau dibikin paragrap \textbf{cukup enter aja}, tidak usah pakai \verb|par| dsb

%\subsection{Soal 2}
%Isi jawaban soal ke-2

%\subsection{Soal 3}
%Isi jawaban soal ke-3

\section{Doli Jonviter}
\subsection{Soal 1}
Isi jawaban soal ke-1

Kalau mau dibikin paragrap \textbf{cukup enter aja}, tidak usah pakai \verb|par| dsb

%\subsection{Soal 2}
%Isi jawaban soal ke-2

%\subsection{Soal 3}
%Isi jawaban soal ke-3

\section{Rahmatul Ridha}
\subsection{Soal 1}
Isi jawaban soal ke-1

Kalau mau dibikin paragrap \textbf{cukup enter aja}, tidak usah pakai \verb|par| dsb

%\subsection{Soal 2}
%Isi jawaban soal ke-2

%\subsection{Soal 3}
%Isi jawaban soal ke-3

\section{Tomy Prawoto}
\subsection{Soal 1}
Isi jawaban soal ke-1

Kalau mau dibikin paragrap \textbf{cukup enter aja}, tidak usah pakai \verb|par| dsb

%\subsection{Soal 2}
%Isi jawaban soal ke-2

%\subsection{Soal 3}
%Isi jawaban soal ke-3
